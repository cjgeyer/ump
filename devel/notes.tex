
\documentclass[11pt]{article}

\usepackage{amsmath}
\usepackage{indentfirst}
\usepackage{xr-hyper}
\usepackage[colorlinks=true,citecolor=blue,urlcolor=blue]{hyperref}

\externaldocument[a-]{algo}

\newcommand{\inner}[1]{\langle #1 \rangle}
\newcommand{\set}[1]{\,\{ #1 \,\}}
\newcommand{\abs}[1]{\lvert #1 \rvert}

\newcommand{\fatdot}{\,\cdot\,}

\newcommand{\opand}{\mathop{\rm and}}
\newcommand{\opor}{\mathop{\rm or}}

\let\code=\texttt

\newcommand{\REVISED}{\begin{center} \LARGE REVISED DOWN TO HERE \end{center}}
\newcommand{\MOVED}[1][equation]{\begin{center} [#1 moved] \end{center}}

\begin{document}

\section{UMPU Dependence on Parameter}

\subsection{Fuzzy $P$-Values}

In calculating fuzzy $P$-values, which are critical functions
$\phi(x, \fatdot, \theta)$ for fixed $x$ and $\theta$,
we use either equation \eqref{a-eq:umpu-spec-a} and \eqref{a-eq:umpu-spec-b}
in the main document (for large values of the argument) or
\eqref{a-eq:gamma-1} and \eqref{a-eq:gamma-2} (for small values of the
argument.

Let $\mu$ be the mean value parameter corresponding to $\theta$.
If $x = \mu$, we use \eqref{a-eq:umpu-spec-b} to obtain the value
of the critical function.
If $x < \mu$, we use \eqref{a-eq:gamma-1} to obtain the value.
If $x > \mu$, we use \eqref{a-eq:gamma-2} to obtain the value.

In case $x = \mu$, we are done.  Equation \eqref{a-eq:umpu-spec-b} gives
the entire distribution for the fuzzy $P$-value.

In case $x < \mu$, we know $C_1 = x$ gives values of the critical function
strictly between zero and one, but we do not yet know the value or values
of $C_2$ that work and need an algorithm to find them.

The case $x > \mu$ is similar to the $x < \mu$ case (just swap left and
right on the number line).  So we only do the former.

Here we repeat the key equations
\eqref{a-eq:gamma-1} and \eqref{a-eq:gamma-2}.
\begin{align}
   \gamma_1
   & =
   \frac{\alpha (C_2 - \mu) + (M_1 - C_2 P_1) + (M_2 - C_2 P_2)}
   {p_1 (C_2 - C_1)}
   \tag{\ref{a-eq:gamma-1}}
   \\
   \gamma_2
   & =
   \frac{\alpha (\mu - C_1) - (M_2 - C_1 P_2) - (M_1 - C_1 P_1)}
   {p_2 (C_2 - C_1)}
   \tag{\ref{a-eq:gamma-2}}
\end{align}
We use these with $C_1 = x$ and a guess for $C_2$.  If both equations
give values between zero and one (inclusive) for some $\alpha$,
then we have a correct $C_2$ (not necessarily the only one).


\REVISED

Subtracting $\mu$ times the first from the second gives
$$
   E_\theta\{ (X - \mu) \phi(X, \alpha, \theta) \} = 0
$$
and this can be rewritten
$$
   E_\theta\{ (\mu - X) I_{(- \infty, \mu)}(X) \phi(X, \alpha, \theta) \}
   =
   E_\theta\{ (X - \mu) I_{(\mu, \infty)}(X) \phi(X, \alpha, \theta) \}
$$
and using $C_1 \le \mu \le C_2$ and the form of $\phi$ given
by the UMPU theorem the latter can be rewritten
\begin{multline*}
   \int_{- \infty}^{C_1} (\mu - x) \phi(x, \alpha, \theta)
   e^{x \theta - c(\theta)} \, \lambda(d x)
   \\
   =
   \int_{C_2}^\infty (x - \mu) \phi(x, \alpha, \theta)
   e^{x \theta - c(\theta)} \, \lambda(d x)
\end{multline*}
where $c(\theta)$ is the cumulant function and $\lambda$ is an order
discrete positive measure (with respect to which the exponential family
has densities).  We note that the integrands on both sides of this
equation are nonnegative and zero only when $x = \mu$.

\subsection{Basic Exponential Family Theory}

If we differentiate
$$
   \frac{\partial}{\partial \theta} \log f_\theta(x) = x - \mu
$$
where $\mu$ is the mean value parameter corresponding to $\theta$
we see that as $\mu$ goes over its range of values (also an open interval
because the $\theta \to \mu$ map is infinitely differentiable and monotone)
that probabilities increase as $\mu$ increases for $x < \mu$ and then decrease.

Here we are interested in the case where the mean value parameter space is
unbounded above, so $\mu \to \infty$ as $\theta$ converges to the upper
bound of its parameter space, which we call $B$, which may be either finite
or infinity.

We have two different arguments depending on whether $B$ is finite or not.

First assume $B$ is infinite, if $x_1 < x_2$,
$$
   \frac{f_\theta(x_1)}{f_\theta(x_2)} = e^{(x_1 - x_2) \theta}
$$
as $\theta \to \infty$ (in this case) the right-hand side goes to $- \infty$.
But (for discrete data) probabilities are bounded by one so
\begin{equation} \label{eq:limit-zero}
   f_\theta(x) \to 0, \qquad \text{as $\theta \to B$ and $\mu \to \infty$}.
\end{equation}
(And there is a similar argument for continuous data involving probabilities
of intervals in the denominator.)

Second assume $B$ is finite, then
$$
   f_\theta(x) \le e^{x B - c(\theta)}
$$
but $c(\theta) \to \infty$ as $\theta \to B$ by lower-semicontinuity
of cumulant functions.  So again we have \eqref{eq:limit-zero}.

Now let $A$ be any event that is bounded above (in the mean value parameter
space, where $X$ also takes values).  Then for any nonnegative function $g$
$$
    E_\mu \{ I_A(X) g(X) \} \to 0, \qquad \text{as $\mu \to \infty$}
$$
by monotone convergence (because the integrand is decreasing after $\mu$
is greater than any element of $A$), provided the expectation exists for
any such $\mu$.

\subsection{Fuzzy Confidence Intervals}

So now we want to prove that in the formula for the UMPU critical function
we have $C_1 \to \infty$ as $\mu \to \infty$.
From what we have already proved
\begin{align*}
   E_\mu\{ I_{(- \infty, \mu)}(X) \phi(X, \alpha, \mu) \} & \to 0
   \\
   E_\mu\{ (\mu - X) I_{(- \infty, \mu)}(X) \phi(X, \alpha, \mu) \} & \to 0
\end{align*}

\begin{center} \LARGE REVISED DOWN TO HERE \end{center}

For a regular full exponential family the canonical parameter space
is an open convex subset of the vector space where that parameter lives.
Hence an open interval if we have a one-parameter family.

Now $\mu \to \infty$ implies
$$
   \int x e^{x \theta - c(\theta)} \, \lambda(d x) \to \infty
$$
but for $x_1 < x_2$ with both possible values of $X$ we have
$$
   \frac{f_\theta(x_1)}{f_\theta(x_2)} = e^{(x_1 - x_2) \theta}
$$

\begin{center} \LARGE REVISED DOWN TO HERE \end{center}

and multiplying both sides through by $e^{c(\theta)}$ gives
$$
   \int_{- \infty}^{C_1} (\mu - x) \phi(x, \alpha, \theta)
   e^{x \theta} \, \lambda(d x)
   =
   \int_{C_2}^\infty (x - \mu) \phi(x, \alpha, \theta)
   e^{x \theta} \, \lambda(d x)
$$
and multiplying both sides through by $e^{x \psi} e^{- x \psi}$ gives
$$
   \int_{- \infty}^{C_1} (\mu - x) \phi(x, \alpha, \theta)
   e^{x (\theta + \psi)} \, \lambda^*(d x)
   =
   \int_{C_2}^\infty (x - \mu) \phi(x, \alpha, \theta)
   e^{x (\theta + \psi)} \, \lambda^*(d x)
$$
where $\lambda^*(d x) = e^{- x \psi} \lambda(d x)$,
and multiplying both sides through by $e^{- \mu (\theta + \psi)}$ gives
\begin{multline*}
   \int_{- \infty}^{C_1} (\mu - x) \phi(x, \alpha, \theta)
   e^{(x - \mu) (\theta + \psi)} \, \lambda^*(d x)
   \\
   =
   \int_{C_2}^\infty (x - \mu) \phi(x, \alpha, \theta)
   e^{(x - \mu) (\theta + \psi)} \, \lambda^*(d x)
   \\
   \ge
   \int_{s(C_2)}^\infty (x - \mu)
   e^{(x - \mu) (\theta + \psi)} \, \lambda^*(d x)
\end{multline*}
where $s(\fatdot)$ denotes the successor function in the ordered set
that is the support of $X$, that is, $s(C_2)$ is the support point next
larger than $C_2$.

And now we have two cases.
It may be that $\mu \to \infty$ implies
$\theta \to \infty$.  This is the case for all of our distributions of
interest except negative binomial.
Or it may be that $\mu \to \infty$ implies $\theta \to \psi$, where $\psi$
is the upper boundary point of the canonical parameter space.  In the case
of the negative binomial with the usual canonical parameter $\psi = 0$.

\section{UMPU Dependence on Parameter}

These are the equations we use to determine the UMPU test,
equations (3.9a) and (3.9b) in the Geyer and Meeden (2005).
\begin{equation} \label{eq:one-minus-gamma}
\begin{split}
   1 - \gamma_1
   & =
   \frac{(1 - \alpha) (C_2 - \mu) + m_{1 2} - C_2 p_{1 2}}
   {p_1 (C_2 - C_1)}
   \\
   1 - \gamma_2
   & =
   \frac{(1 - \alpha) (\mu - C_1) - m_{1 2} + C_1 p_{1 2}}
   {p_2 (C_2 - C_1)}
\end{split}
\end{equation}
Alternatively, we have
\begin{equation} \label{eq:gamma}
\begin{split}
   \gamma_1
   & =
   \frac{\alpha (C_2 - \mu) + (M_1 - C_2 P_1) + (M_2 - C_2 P_2)}
   {p_1 (C_2 - C_1)}
   \\
   \gamma_2
   & =
   \frac{\alpha (\mu - C_1) - (M_2 - C_1 P_2) - (M_1 - C_1 P_1)}
   {p_2 (C_2 - C_1)}
\end{split}
\end{equation}
which are (12a) and (12b) in \code{algo.pdf} in this repository,
and the quantities therein are defined in the referenced documents.
%
% Check both of these in Mathematica
%
% gamma1 = 1 - ((1 - alpha) (C2 - mu) + m12 - C2 p12) / (p1 (C2 - C1))
% gamma2 = 1 - ((1 - alpha) (mu - C1) - m12 + C1 p12) / (p2 (C2 - C1))
% (1 - p1 - p2 - p12) + gamma1 p1 + gamma2 p2
% Simplify[%]
% (mu - p1 C1 - p2 C2 - m12) + gamma1 p1 C1 + gamma2 p2 C2
% Simplify[%]
%
% gamma1 = (alpha (C2 - mu) + (M1 - C2 P1) + (M2 - C2 P2)) / (p1 (C2 - C1))
% gamma2 = (alpha (mu - C1) - (M2 - C1 P2) - (M1 - C1 P1)) / (p2 (C2 - C1))
% P1 + P2 + p1 gamma1 + p2 gamma2
% Simplify[%]
% M1 + M2 + p1 gamma1 C1 + p2 gamma2 C2
% Simplify[%]

In order to avoid writing pairs of equations we take $i$ to be either 1 or 2
and $j$ to be the other of 1 or 2 (that is, $j = 3 - i$).
We also take the canonical statistic to be $X$ rather than $T(X)$.
We also define the events
\begin{align*}
   A_1 & = \set{ x : x < C_1 }
   \\
   A_2 & = \set{ x : C_2 < x }
   \\
   A_{1 2} & = \set{ x : C_1 < x < C_2 }
\end{align*}
and write $A_{2 1} = A_{1 2}$, $m_{2 1} = m_{1 2}$, and
$p_{2 1} = p_{1 2}$.  Then \eqref{eq:one-minus-gamma} becomes
\begin{align*}
   1 - \gamma_i
   & =
   \frac{(1 - \alpha) (C_j - \mu) + m_{i j} - C_j p_{i j}}
   {p_i (C_j - C_i)}
   \\
   & =
   \frac{ E_\theta \left\{ (1 - \alpha) (C_j - X) + (X - C_j) I_{A_{1 2}}(X)
   \right\}}
   {f_\theta(C_i) (C_j - C_i)}
   \\
   & =
   \frac{ E_\theta \left\{ [I_{A_{1 2}^c}(X) - \alpha] (C_j - X) \right\}}
   {f_\theta(C_i) (C_j - C_i)}
   \\
   & =
   \int
   \frac{[I_{A_{1 2}^c}(x) - \alpha] (C_j - x)}
   {C_j - C_i} \cdot \frac{f_\theta(x)}{f_\theta(C_i)} \lambda(d x)
   \\
   & =
   \int
   \frac{[I_{A_{1 2}^c}(x) - \alpha] (C_j - x)}
   {C_j - C_i} \cdot e^{\inner{x - C_i, \theta}} \lambda(d x)
\end{align*}
So
\begin{equation} \label{eq:deriv}
   - \frac{\partial \gamma_i}{\partial \theta} =
   \int
   \frac{[I_{A_{1 2}^c}(x) - \alpha] (C_j - x) (x - C_i)}
   {C_j - C_i} \cdot e^{\inner{x - C_i, \theta}} \lambda(d x)
\end{equation}

Take the case $i = 1$ and $j = 2$.
For $x \in A_{1 2}$ we have
\begin{gather*}
   I_{A_{1 2}^c}(x) - \alpha < 0
   \\
   C_i < x < C_j
\end{gather*}
so the integrand in \eqref{eq:deriv} is negative.
For $x \notin A_{1 2}$ we have
\begin{gather*}
   I_{A_{1 2}^c}(x) - \alpha > 0
   \\
   x \le C_i < C_j \opor C_i < C_j \le x
\end{gather*}
hence
$$
   (C_j - x) (x - C_i)  \le 0
$$
so the integrand in \eqref{eq:deriv} is nonpositive.
Hence \eqref{eq:deriv} is negative,
and $\partial \gamma_1 / \partial \theta$ is positive.

Now take the case (the other equation) $i = 2$ and $j = 1$.
For $x \in A_{1 2}$ we have
\begin{gather*}
   I_{A_{1 2}^c}(x) - \alpha < 0
   \\
   C_j < x < C_i
\end{gather*}
so the integrand in \eqref{eq:deriv} is positive.
For $x \notin A_{1 2}$ we have
\begin{gather*}
   I_{A_{1 2}^c}(x) - \alpha > 0
   \\
   x \le C_j < C_i \opor C_j < C_i \le x
\end{gather*}
hence
$$
   (C_j - x) (x - C_i)  \le 0
$$
so the integrand in \eqref{eq:deriv} is nonnegative.
Hence \eqref{eq:deriv} is positive,
and $\partial \gamma_1 / \partial \theta$ is negative.

\subsection{Try II}

Want to show that in a regular full exponential family, if the mean value
parameter $\mu$ goes to infinity, then
$$
   E_\mu \left\{ \abs{X} I_{(- \infty, b)}(X) \right\} \to 0
$$
for any fixed $b$.

Write
\begin{align*}
   E_\mu \left\{ \abs{X} I_{(- \infty, b)}(X) \right\}
   & =
   \int_{- \infty}^b \abs{x} f_\theta(x) \lambda(d x)
   \\
   & =
   e^{- c(\theta)} \int_{- \infty}^b \abs{x} e^{x \theta} \lambda(d x)
\end{align*}

\subsection{Try I}
\end{document}
