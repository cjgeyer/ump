
\documentclass[11pt]{article}

\usepackage{amsmath}
\usepackage{indentfirst}

\newcommand{\inner}[1]{\langle #1 \rangle}
\newcommand{\set}[1]{\,\{ #1 \,\}}

\newcommand{\opand}{\mathop{\rm and}}
\newcommand{\opor}{\mathop{\rm or}}

\let\code=\texttt

\begin{document}

\section{UMPU Dependence on Parameter}

These are the equations we use to determine the UMPU test,
equations (3.9a) and (3.9b) in the Geyer and Meeden (2005).
\begin{equation} \label{eq:one-minus-gamma}
\begin{split}
   1 - \gamma_1
   & =
   \frac{(1 - \alpha) (C_2 - \mu) + m_{1 2} - C_2 p_{1 2}}
   {p_1 (C_2 - C_1)}
   \\
   1 - \gamma_2
   & =
   \frac{(1 - \alpha) (\mu - C_1) - m_{1 2} + C_1 p_{1 2}}
   {p_2 (C_2 - C_1)}
\end{split}
\end{equation}
Alternatively, we have
\begin{equation} \label{eq:gamma}
\begin{split}
   \gamma_1
   & =
   \frac{\alpha (C_2 - \mu) + (M_1 - C_2 P_1) + (M_2 - C_2 P_2)}
   {p_1 (C_2 - C_1)}
   \\
   \gamma_2
   & =
   \frac{\alpha (\mu - C_1) - (M_2 - C_1 P_2) - (M_1 - C_1 P_1)}
   {p_2 (C_2 - C_1)}
\end{split}
\end{equation}
which are (12a) and (12b) in \code{algo.pdf} in this repository,
and the quantities therein are defined in the referenced documents.
%
% Check both of these in Mathematica
%
% gamma1 = 1 - ((1 - alpha) (C2 - mu) + m12 - C2 p12) / (p1 (C2 - C1))
% gamma2 = 1 - ((1 - alpha) (mu - C1) - m12 + C1 p12) / (p2 (C2 - C1))
% (1 - p1 - p2 - p12) + gamma1 p1 + gamma2 p2
% Simplify[%]
% (mu - p1 C1 - p2 C2 - m12) + gamma1 p1 C1 + gamma2 p2 C2
% Simplify[%]
%
% gamma1 = (alpha (C2 - mu) + (M1 - C2 P1) + (M2 - C2 P2)) / (p1 (C2 - C1))
% gamma2 = (alpha (mu - C1) - (M2 - C1 P2) - (M1 - C1 P1)) / (p2 (C2 - C1))
% P1 + P2 + p1 gamma1 + p2 gamma2
% Simplify[%]
% M1 + M2 + p1 gamma1 C1 + p2 gamma2 C2
% Simplify[%]

In order to avoid writing pairs of equations we take $i$ to be either 1 or 2
and $j$ to be the other of 1 or 2 (that is, $j = 3 - i$).
We also take the canonical statistic to be $X$ rather than $T(X)$.
We also define the events
\begin{align*}
   A_1 & = \set{ x : x < C_1 }
   \\
   A_2 & = \set{ x : C_2 < x }
   \\
   A_{1 2} & = \set{ x : C_1 < x < C_2 }
\end{align*}
and write $A_{2 1} = A_{1 2}$, $m_{2 1} = m_{1 2}$, and
$p_{2 1} = p_{1 2}$.  Then \eqref{eq:one-minus-gamma} becomes
\begin{align*}
   1 - \gamma_i
   & =
   \frac{(1 - \alpha) (C_j - \mu) + m_{i j} - C_j p_{i j}}
   {p_i (C_j - C_i)}
   \\
   & =
   \frac{ E_\theta \left\{ (1 - \alpha) (C_j - X) + (X - C_j) I_{A_{1 2}}(X)
   \right\}}
   {f_\theta(C_i) (C_j - C_i)}
   \\
   & =
   \frac{ E_\theta \left\{ [I_{A_{1 2}^c}(X) - \alpha] (C_j - X) \right\}}
   {f_\theta(C_i) (C_j - C_i)}
   \\
   & =
   \int
   \frac{[I_{A_{1 2}^c}(x) - \alpha] (C_j - x)}
   {C_j - C_i} \cdot \frac{f_\theta(x)}{f_\theta(C_i)} \lambda(d x)
   \\
   & =
   \int
   \frac{[I_{A_{1 2}^c}(x) - \alpha] (C_j - x)}
   {C_j - C_i} \cdot e^{\inner{x - C_i, \theta}} \lambda(d x)
\end{align*}
So
\begin{equation} \label{eq:deriv}
   - \frac{\partial \gamma_i}{\partial \theta} =
   \int
   \frac{[I_{A_{1 2}^c}(x) - \alpha] (C_j - x) (x - C_i)}
   {C_j - C_i} \cdot e^{\inner{x - C_i, \theta}} \lambda(d x)
\end{equation}

Take the case $i = 1$ and $j = 2$.
For $x \in A_{1 2}$ we have
\begin{gather*}
   I_{A_{1 2}^c}(x) - \alpha < 0
   \\
   C_i < x < C_j
\end{gather*}
so the integrand in \eqref{eq:deriv} is negative.
For $x \notin A_{1 2}$ we have
\begin{gather*}
   I_{A_{1 2}^c}(x) - \alpha > 0
   \\
   x \le C_i < C_j \opor C_i < C_j \le x
\end{gather*}
hence
$$
   (C_j - x) (x - C_i)  \le 0
$$
so the integrand in \eqref{eq:deriv} is nonpositive.
Hence \eqref{eq:deriv} is negative,
and $\partial \gamma_1 / \partial \theta$ is positive.

Now take the case (the other equation) $i = 2$ and $j = 1$.
For $x \in A_{1 2}$ we have
\begin{gather*}
   I_{A_{1 2}^c}(x) - \alpha < 0
   \\
   C_j < x < C_i
\end{gather*}
so the integrand in \eqref{eq:deriv} is positive.
For $x \notin A_{1 2}$ we have
\begin{gather*}
   I_{A_{1 2}^c}(x) - \alpha > 0
   \\
   x \le C_j < C_i \opor C_j < C_i \le x
\end{gather*}
hence
$$
   (C_j - x) (x - C_i)  \le 0
$$
so the integrand in \eqref{eq:deriv} is nonnegative.
Hence \eqref{eq:deriv} is positive,
and $\partial \gamma_1 / \partial \theta$ is negative.

\end{document}
